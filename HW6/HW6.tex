\documentclass[11pt]{article}
\usepackage{../EllioStyle}
\usepackage{fancyvrb}
\usepackage{pdfpages}
\usepackage{xcolor}


\usepackage{fancyvrb}

% redefine \VerbatimInput
\RecustomVerbatimCommand{\VerbatimInput}{VerbatimInput}%
{fontsize=\footnotesize,
 %
 frame=lines,  % top and bottom rule only
 framesep=2em, % separation between frame and text
 rulecolor=\color{gray},
 %
 label=\fbox{\color{black}design pattern finder output.txt},
 labelposition=topline,
 %
 commandchars=\|\(\), % escape character and argument delimiters for
                      % commands within the verbatim
 commentchar=*        % comment character
}


\title{Homework 6}
\author{Elliott Pryor}
\date{19 Nov 2020}


\begin{document}
\maketitle


\problem{1}
This system is designed to be as simple as possible. Working on the KISS principles. It prioritizes flexible methods for user interaction: web, mobile, or future technologies. 
The system is designed to interact with a variety of platforms in a comparable manner, giving users an equal experience regardless of platform.
It also prioritizes marketing allowing the business to create one uniform marketing strategy to roll out across all transactions.
The B2C and B2B interactions are self contained making it easier to change other components in the system (low coupling). 
It is built on one back-end system so that all processes can have high quality security and uniform record keeping. 

\begin{figure}[H]
    \centering
    \includegraphics[scale = 0.7]{./p1_high_level.png}
    \caption{High Level diagram of the system}
    \label{fig:high level}
\end{figure}

\newpage
\problem{2}
\begin{figure}[H]
    \centering
    \includegraphics[width = 0.8\linewidth]{./p2_component.png}
    \caption{Component diagram}
    \label{fig:component}
\end{figure}

At the top level we have the user interaction systems. 
The display component is the centralized system that compiles the relevant information for Mobile and Web to use in their displays.
They take this information through the disp interface and use it to create their unique interaction. 
This is designed so it can be expanded to more platforms easily. 
The display component collects information from the business to customer and business to business components as well as the overall marketing strategy to curate content. 
It also provides UI interface which is the user input from the end display, such as credentials in login, or searchbar, or clicks/taps. 
The B2C and B2B components handle all the processes related to business to customer and business to business relations. 
They provide interfaces to share this information with Display or other components. Security provides one backend for secure processes. 
It uses the UI to gather login credentials and verify against customers (cust). 
It authenticates logins and can use order history or other methods for fraud detection or credit checking for B2B operations. 
Inventory provides a set of objects that we are selling, and also methods for logging transactions and order histories. 
Marketing is its own component. It is designed to give a uniform marketing strategy to be deployed across the company to strengthen overall ad/marketing campaigns. 

\newpage
\problem{3}

This takes user inputs in and manages them with the Organization method. The organization method can lookup user credentials and find the proper user. Then it can apply changes made in the UI of choice (web/mobile) to the customer object. Like add a searched object in to a wishlist, or connect a new social media account. The social media accounts use the strategy pattern to allow connecting many accounts. It is also easy to expand the business and add a new account type. 

\begin{figure}[H]
    \centering
    \includegraphics[width = \linewidth]{./p3_uml.png}
    \caption{UML Diagram of B2C component}
    \label{fig:uml}
\end{figure}

\end{document}